\documentclass{article}
\usepackage[utf8]{inputenc}
\usepackage[english]{babel}
 
\setlength{\parindent}{4em}
\setlength{\parskip}{1em}
\usepackage{xcolor}
\usepackage{latexsym}
\usepackage{graphicx}
\usepackage{subcaption}
\usepackage[margin=1.2in]{geometry}
\usepackage{parskip}
\usepackage{braket}
\usepackage{physics}
\usepackage{amsmath}
\usepackage{amssymb}


\title{Sum Over Unitaries}
\author{Logan Xu}
\date{May 16, 2019}

\begin{document}
\maketitle

\section{Amplitude Amplification}
Given a known state $\ket{\psi}$ in a Hilbert space $\mathcal{H}$, and two mutially orthogonal subspaces, the 'good' subspace $\mathcal{H}_G$ and the 'bad' subspace $\mathcal{H}_B$, our goal is to evolve the initial state $\ket{\psi}$ into $\mathcal{H}_G$. We define a Hermitian $P_G$ to be the projection operator $\mathcal{H}\rightarrow \mathcal{H}_G$. Then we have
\begin{equation}\label{1}
    \ket{\psi} = P_G\ket{\psi} + (1-P_G)\ket{\psi} = \sin{\theta}\ket{\psi_G} + \cos{\theta}\ket{\psi_B},
\end{equation}
where $\theta = \arcsin{(\abs{P_G\ket{\psi}})}$, and $\ket{\psi_G}$ and $\ket{\psi_B}$ are the normalized projections of $\ket{\psi}$ into $\mathcal{H}_G$ and $\mathcal{H}_B$. Here notice that the probability of finding the system in 'good' state is $\sin^2{\theta}$.\\
Further define
\begin{equation}\label{2}
    \ket{\psi ^\bot} = \cos({\theta}) \ket{\psi_G} - \sin({\theta}) \ket{\psi_B},
\end{equation}
such that $\braket{\psi ^\bot}{\psi} = 0$.\\
Multiply equation \ref{1} by $\sin{\theta}$ and equation \ref{2} by $\cos{\theta}$, we get
\begin{equation}\label{3}
    \sin({\theta}) \ket{\psi} = \sin^2({\theta}) \ket{\psi_G} + \sin({\theta}) \cos({\theta}) \ket{\psi_B},
\end{equation}
\begin{equation}\label{4}
    \cos({\theta}) \ket{\psi ^\bot} = \cos^2({\theta}) \ket{\psi_G} + \sin({\theta}) \cos({\theta}) \ket{\psi_B}.
\end{equation}
Adding equation \ref{3} and equation \ref{4}, we get
\begin{equation}
    \ket{\psi_G} = \sin({\theta}) \ket{\psi} + \cos({\theta}) \ket{\psi ^\bot}.
\end{equation}
Obviously, the 'bad' state can de written as
\begin{equation}
    \ket{\psi_B} = \cos({\theta}) \ket{\psi} - \sin({\theta}) \ket{\psi ^\bot}.
\end{equation}
We define a new projection operator $P = \ket{\psi}\bra{\psi}$, therefore
\begin{equation}
    P\ket{\psi_G} = \sin({\theta})P\ket{\psi} + \cos({\theta})P\ket{\psi ^\bot} = \sin({\theta})\ket{\psi}.
\end{equation}
\begin{equation}
    P\ket{\psi_B} = \cos({\theta})P\ket{\psi} - \sin({\theta})P\ket{\psi ^\bot} = \cos({\theta})\ket{\psi}.
\end{equation}
Define a new unitary operator $Q = -S_\psi S_G$, where
\begin{equation}
    S_\psi = 1 - 2P, \nonumber
\end{equation}
\begin{equation}
    S_G = 1 - 2P_G. \nonumber
\end{equation}
$S_\psi$ flips the phase of the initial state, and $S_G$ flips the phase in the 'good' subspace. Now we have
\begin{equation}
    Q\ket{\psi_G} = -S_\psi S_G \ket{\psi_G} = (1 - 2P) \ket{\psi_G} = \ket{\psi_G} - 2\sin({\theta})\ket{\psi} = \cos({2\theta})\ket{\psi_G} - \sin({2\theta})\ket{\psi_B}
\end{equation}
\begin{equation}
    Q\ket{\psi_B} = -S_\psi S_G \ket{\psi_B} = -(1 - 2P) \ket{\psi_B} = - \ket{\psi_B} + 2\cos({\theta}) \ket{\psi_G} = \sin({2\theta})\ket{\psi_G} + \cos({2\theta})\ket{\psi_B}.
\end{equation}
Thus in the $\ket{\psi_G}$, $\ket{\psi_B}$ basis we have
\begin{equation}
    Q = 
\begin{pmatrix}

\cos({2\theta})&\sin({2\theta})\\
-\sin({2\theta})&\cos({2\theta})
    
\end{pmatrix}.
\end{equation}
It is obvious that applying the $Q$ operator $n$ times on the state $\ket{\psi}$ gives 
\begin{equation}
    Q^n \ket{\psi} = \sin({(2n+1)\theta})\ket{\psi_G} + \cos({(2n+1)\theta})\ket{\psi_B}.
\end{equation}
By applying $Q$ to our state for $n$ times (strategically chosen), we might be able to amplify the amplitude of $\ket{\psi_G}$ thus increase the probability of measuring the state in the 'good' subspace. 

\section{Grover's Algorithm}
We are given a task to find out $M$ solutions to a problem amongst an unsorted database containing $N$ elements ($N\leqslant M$). Suppose there exists an oracle $O$ that is able to tell whenever we find a solution. Precisely, the oracle is unitary operator that will flip a qubit whenever it sees a solution ($f(x)=1$), and do nothing if it is not a solution ($f(x)=0$).
\begin{equation}
    O\ket{x} = (-1)^{f(x)} \ket{x}.
\end{equation}
We define another unitary operator $S_ \psi = 2P - 1$, where the projection operator $P := \ket{\psi} \bra{\psi}$, and the Grover's algorithm is given as following: \\
\begin{enumerate}
    \item Initialize the system to the state (by applying $H^{\otimes n}$)
    \begin{equation}
        \ket{\psi} = \frac{1}{\sqrt{N}} \sum_{x=0}^{N-1}\ket{x} \nonumber
    \end{equation}
    \item Perform the following subroutine, $G$, $r(N)$ times, which is asymptotically $O(\sqrt{N})$
    \begin{enumerate}
        \item Apply operator $O$
        \item Apply operator $S_ \psi$
    \end{enumerate}
    \item Perform measurement.
\end{enumerate}
Shown in figure \ref{grovers}, the subroutine $G$ can be seen as a series rotations in the 2-D space spanned by $\ket{\alpha}$ and $\ket{\beta}$. Initially the state $\ket{\psi}$ is inclined at angle $\theta /2$ from $\ket{\alpha}$. The oracle $O$ reflects the state about the state $\ket{\alpha}$, then the operator $S_ \psi$ reflects it about $\ket{\psi}$. By performing such a subroutine $r(N)$ times, we can make the probability of finding the state $\ket{\psi}$ in $\ket{\beta}$ approaching $1$.\\
Note here we say the oracle $O$ reflects the state about $\ket{\alpha}$. If we rename our 'good' subspace as $\ket{\beta}$ and 'bad' subspace as $\ket{\alpha}$, the oracle $O$ becomes $1 - 2P_G$, i.e. reflection about the 'bad' subspace, and the subroutine $G$ then becomes $G = (2P - 1)(1 - 2P_G) = -(1 - 2P)(1 - 2P_G)$, same as our $Q$ operator defined in the preceding section.
\begin{figure}[h]
    \centering
    \includegraphics[width=10cm,height=10cm] {grovers.png}
    \caption{Geometrically interpretation of what the first iteration of the subroutine does}
    \label{grovers}
\end{figure}

\section{Oblivious Amplitude Amplification}
The conventional amplitude amplification relies on using the projection operator $P_\psi$ which is not possible if we do not know $\ket{\psi}$. Oblivious amplitude amplification can be performed to achieve similar goal even when the initial state $\ket{\psi}$ is unknown.\\
Assume there exists a unitary operator $W$, such that
\begin{equation}\label{14}
    W\ket{0}\ket{\psi} = \frac{1}{s}\ket{0}U\ket{\psi} + \sqrt{1 - \frac{1}{s^2}}\ket{\Phi ^\bot},
\end{equation}
where $U = \sum_j \beta_j V_j$, $s = \sum_j \beta_j$, $\ket{0}$ represents the ancilla with all states in the 0 state, and $\ket{\Phi ^\bot}$ is a state whose ancillary state is orthogonal to $\ket{0}$. Here $U$ is derived from the time evolution operator $exp(-iHt)$ (not unitary) by rewriting it in Taylor series $\sum_{k=0}^K \frac{1}{k!}(-iHt/r)^k$ (not unitary). We use the form 
\begin{equation}
    H = \sum_{l=1}^L \alpha_l H_l \nonumber,
\end{equation}
where each $H_l$ is unitary, to rewrite the Taylor series as
\begin{equation}
    U = \sum_j \beta_j V_j,
\end{equation}
where we have reached our goal that each $V_j$ is unitary, and $\beta_j > 0 $. Define
\begin{equation}
    \ket{\Psi} = \ket{0}\ket{\psi}, \ket{\Phi} = \ket{0}U\ket{\psi}, \sin(\theta) = \frac{1}{s}.
\end{equation}
Then equation \ref{14} becomes
\begin{equation}
    W\ket{\Psi} = \sin(\theta)\ket{\Phi} + \cos(\theta) \ket{\Phi ^\bot}.
\end{equation}
It is obvious that
\begin{equation}
    W\ket{\Psi ^\bot} = \cos(\theta)\ket{\Phi} - \sin(\theta) \ket{\Phi ^\bot}.
\end{equation}
Let $P = \ket{0}\bra{0} \otimes 1$, so $P \ket{\Phi ^\bot} = 0$ and $P \ket{\Psi ^\bot} = 0$. The former is true by definition, and the latter is proved by Childs' group shown in their paper as \textbf{Lemma 3.7 (2D Subspace Lemma)}. Using the same trick as we did in the first section, we obtain
\begin{equation}
    W^\dagger \ket{\Phi} = \sin(\theta)\ket{\Psi} + \cos(\theta) \ket{\Psi ^\bot},
\end{equation}
\begin{equation}
    W^\dagger \ket{\Phi ^\bot} = \cos(\theta)\ket{\Psi} - \sin(\theta) \ket{\Psi ^\bot}.
\end{equation}
Further define $R = 2P - 1$ and $S = -WRW^\dagger R$ as we did similarly in the first section, we obtain
\begin{equation}
    S\ket{\Phi} = -WRW^\dagger \ket{\Phi} = -W(\sin(\theta)\ket{\Psi} - \cos(\theta)\ket{\Psi ^\bot}) = \cos(2 \theta)\ket{\Phi} - \sin(2 \theta)\ket{\Phi},
\end{equation}
\begin{equation}
    S\ket{\Phi ^\bot} = WRW^\dagger \ket{\Phi ^\bot} = W(\cos(\theta)\ket{\Psi} + \sin(\theta)\ket{\Psi ^\bot}) = \sin(2 \theta)\ket{\Phi} + \cos(2 \theta)\ket{\Phi}.
\end{equation}
Therefore
\begin{equation}
    S = 
\begin{pmatrix}

\cos({2\theta})&\sin({2\theta})\\
-\sin({2\theta})&\cos({2\theta})
    
\end{pmatrix}.
\end{equation}
It is worth noticing that $S$ acts as a rotation by $2 \theta$ in the subspace spanned by $\ket{\Phi}$ and $\ket{\Phi}$. For any integer $n$,
\begin{equation}\label{24}
    S^n W\ket{\Psi} = \sin((2n+1)\theta)\ket{\Phi} + \cos((2n+1)\theta)\ket{\Phi ^\bot}.
\end{equation}
When $n=1$, equation \ref{24} becomes
\begin{equation}
    SW\ket{\Psi} = \sin(3 \theta)\ket{\Phi} + \cos(3 \theta)\ket{\Phi ^\bot}.
\end{equation}
If we then set $\sin(\theta) = 1/2 \rightarrow \sin(3 \theta) = 0$, $SW$ operator can be performed exactly with $100\%$ success rate in a single step ($n=1$) without having to know the initial state $\ket{\psi}$. This corresponds to the condition $s=2$, and the truncated Taylor series $U = \sum_j \beta_j V_j$ is unitary. \\
For the general case, where the truncated Taylor series of time evolution operator is not unitary, and $s \neq 1$, we have
\begin{equation}
    PSW = \ket{0}(\frac{3}{s}U -\frac{4}{s^3}UU^\dagger U)\ket{\psi}.
\end{equation}

\end{document}
